\large\noindent\fbox{
	\parbox{\textwidth}{
	Utilizzare l'Algoritmo 3.6 del libro per stabilire se le seguenti matrici sono sdp o no,
	\begin{center}
		$A_{1} = \begin{bmatrix}
				1 & -1 & 2 & 2\\
				-1 & 5 & -14 & 2\\
				2 & -14 & 42 & 2\\
				2 & 2 & 2 & 65
			\end{bmatrix}$,
			$A_{2} = \begin{bmatrix}
				1 & -1 & 2 & 2\\
				-1 & 6 & -17 & 3\\
				2 & -17 & 48 & -16\\
				2 & 3 & -16 & 4
			\end{bmatrix}$
	\end{center}
	}
}
\begin{flushleft}
	\large \textbf{Soluzione}\\[0.5cm]
	L'algoritmo utilizzato per risolvere l'esercizio è quello della fattorizzazione $LDL^{T}$. Questo perchè grazie al Teorema 3.6 del libro, è noto che una matrice è sdp se e solo se questa è fattorizzabile $LDL^{T}$.\\Si riporta dunque il codice Matlab di tale fattorizzazione:
	\lstinputlisting[language=Matlab]{Codici/Cap3/LDLTFatt.m}
	$\bullet A_{1}$\\[0.3cm]
	Si richiama il precedente Algoritmo con la matrice $A_{1}$ come input. Si ottiene il seguente risultato:
	\begin{center}
		$A_{1} = \begin{bmatrix}
				1 & -1 & 2 & 2\\
				-1 & 4 & -14 & 2\\
				2 & -3 & 2 & 2\\
				2 & 1 & 5 & 7
			\end{bmatrix}$
		\end{center}
		\newpage
		$\bullet A_{2}$\\[0.3cm]
	Si richiama il precedente Algoritmo con la matrice $A_{2}$ come input. In questo caso però la matrice non risulta essere fattorizzabile $LDL^{T}$ e dunque l'esecuzione stamperà il seguente errore:\\[0.4cm]
		\textbf{Error using LDLTFatt (line 19)}\\	
		\textbf{Matrice non sdp}\\
	\end{flushleft}