\large\noindent\fbox{
	\parbox{\textwidth}{
	Sia $A = \begin{bmatrix}
				\epsilon & 1\\
				1 & 1
			\end{bmatrix}$ con $\epsilon = 10^{-13}$. Definire L triangolare inferiore a diagonale unitaria e U triangolare superiore in modo che il prodotto LU sia la fattorizzazione LU di A e, posto b = Ae, con $e = (1,1)^{T}$, confrontare l'accuratezza della soluzione che si ottiene usando il comando $U \backslash(L \backslash b)$ (Gauss senza pivoting) e il comando $A \backslash b $(Gauss con pivoting),
	}
}
\begin{flushleft}
	\large \textbf{Soluzione}\\[0.5cm]
	Con il seguente codice si effettua la fattorizzazione A = LU:
	\lstinputlisting[language=Matlab]{Codici/Cap3/luFactorization.m}
	Si richiama inoltre il seguente codice Matlab per effettuare la chiamata alla funzione $luFactorization(A)$, ricavare quindi la sottomatrice triangolare inferiore a diagonale unitaria (L) e la sottomatrice triangolare superiore (U); si assegna a b la quantità Ae ed infine si confrontano i risultati usando prima Gauss senza pivoting e poi Gauss con pivoting.
	\lstinputlisting[language=Matlab]{Codici/Cap3/SoluzioneEs6Cap3.m}
	Dalla fattorizzazione si ottiene:\\[0.5cm]
	\begin{center}
	$A = \begin{bmatrix}
				1.0000e-13 & 1\\
				1.0000e+13 & -1.0000e+13
			\end{bmatrix}$,
			$L = \begin{bmatrix}
				1& 0\\
				1.0000e+13 & 1
			\end{bmatrix}$,
			$U = \begin{bmatrix}
				1.0000e-13 & 1\\
				0 & -1.0000e+13
			\end{bmatrix}$\\[0.5cm]
			\end{center}
			Calcolando il vettore b = Ae si ottiene:\\[0.5cm]
			\begin{center}
			$ b = \begin{bmatrix}
				1.000000000000100\\
				1
			\end{bmatrix}$\\[0.5cm]
			Infine si ottengono i due vettori rispettivamente di Gauss senza pivoting e Gauss con pivoting:\\[0.5cm]
			$ gSp = \begin{bmatrix}
				0\\
				1.000000000000100
			\end{bmatrix}$, 
			$ gp = \begin{bmatrix}
				1\\
				1
			\end{bmatrix}$
			\end{center}
			Dunque si nota come Gauss con pivoting sia piu accurato rispetto a Gauss senza pivoting.
	\end{flushleft}
	