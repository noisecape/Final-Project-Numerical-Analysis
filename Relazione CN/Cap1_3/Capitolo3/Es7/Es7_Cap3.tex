\large\noindent\fbox{
	\parbox{\textwidth}{
	Scrivere la function Matlab specifica per la risoluzione di un sistema lineare con matrice dei coefficienti $A \in R^{nxn}$ bidiagonale inferiore a diagonale unitaria di Toeplitz, specificabile con uno scalare $\alpha$. Sperimentarne e commentarne le prestazioni (considerare il numero di condizionamento della matrice) nel caso in cui $n$ = 12 e $\alpha = 100$ ponendo dapprima $b = (1,101,...101)^{T}$ (soluzione esatta $\hat{x} = (1,...,1)^{T})$ e quindi b = 0.1 * $(1,101,...,101)^{T})$ (soluzione esatta $\hat{x} = (0.1,...,0.1)^{T}$).
	}
}
\begin{flushleft}
	\large \textbf{Soluzione}\\[0.5cm]
	Si riporta in seguito il codice utilizzato per la risoluzione di un sistema lineare Ax = b, con $A \in R^{nxn}$ bidiagonale inferiore a diagonale unitaria di Toeplitz, che ha la seguente forma:
	\begin{center}
	$A = \begin{bmatrix}
				1 & 0 & \ldots & \ldots & 0\\
				\alpha & 1 & 0 & \ldots & 0\\
				0 & \ddots & \ddots & \ddots & \vdots\\
				\vdots & \ddots & \ddots & \ddots & \vdots\\
				0 & \ldots & \ldots & \alpha & 1
			\end{bmatrix}$\\[0.5cm]
	\end{center}
	\lstinputlisting[language=Matlab]{Codici/Cap3/trisolveInfAlphaToeplitz.m}
	Il seguente codice invece riguarda la soluzione dell'esercizio:
	\lstinputlisting[language=Matlab]{Codici/Cap3/SoluzioneEs7Cap3.m}
	\newpage
	Si ottengono quindi i due vettori soluzione:\\[0.5cm]
	\begin{center}
	$x = \begin{bmatrix}
				1\\
				1\\
				1\\
				1\\
				1\\
				1\\
				1\\
				1\\
				1\\
				1\\
				1\\
				1\\
			\end{bmatrix}$,
			$x1 = \begin{bmatrix}
				1.0000000000000000e-01\\
				1.0000000000000014e-01\\
				9.999999999985931e-02\\
				1.000000000140702e-01\\
				9.999999859298470e-02\\
				1.000001407015319e-01\\
				9.998592984681665e-02\\
				1.014070153183369e-01\\
				-4.070153183368319e-02\\
				1.417015318336832e+01\\
				-1.406915318336832e+03\\
				1.407016318336832e+05\\
			\end{bmatrix}$
	\end{center}
	Analizzando il condizionamento si nota che la matrice risulta essere mal condizionata, in quanto $condA = 1.0202e+24$ ovvero $k(A) >> 1$.\\
	Ricordiamo che l'errore commesso sulla soluzione risulta:
	\begin{center}
		$\frac{\|\triangle x\|}{\|x\|} \leq \|A\| \cdot \|A^{-1}\| (\frac{\|\triangle b\|}{\|b\|}+\frac{\|\triangle A\|}{\|A\|})$.
	\end{center}
	Considerando l'errore commesso sulla soluzione nel primo caso ($b = (1,101,...101)^{T}$) si ha che il vettore risultato coincide con il vettore esatto che viene fornito dal testo dell'esercizio. Di conseguenza l'errore commesso è pari a zero. Nel secondo caso invece (b = 0.1 * $(1,101,...,101)^{T})$) la soluzione non coincide con la soluzione esatta fornita dal testo; calcolando l'errore sul risultato, questo risulta essere amplificato dal mal condizionamento della matrice:
	\begin{center}
	$errX = 1.7943e+08$
	\end{center}
\end{flushleft}