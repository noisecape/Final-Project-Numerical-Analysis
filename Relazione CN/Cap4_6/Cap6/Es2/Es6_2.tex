\begin{center}
\large\noindent\fbox{
	\parbox{\textwidth}{
    Utilizzare il metodo delle potenze per calcolarne l'autovalore dominante della matrice $A_n$ del precedente esercizio, con una approssimazione $tol=10^{-5}$, partendo da un vettore con elementi costanti. Riempire, quindi, la seguente tabella: \\ \begin{center}
  \begin{tabular}{ | l | c | r |}
    \hline
    $n$ & \textit{numero iterazioni effettuate} & \textit{stima autovalore} \\ \hline
    100 &  &  \\ 
    200 &  &  \\ 
    \vdots &  &  \\  
	1000 &  &  \\
    \hline
  \end{tabular}
\end{center}
}
}\end{center}
\noindent Di seguito sono riportati i codici implementati. La funzione \textit{potenze} calcola sia l'autovalore dominante della matrice \textit{A} , sia il numero di iterazioni, \textit{numIt }, impiegate.
\vspace{0.5cm}

\lstinputlisting[language=Matlab]{Codici/Cap6/SoluzioneEs2_Cap6.m}
\lstinputlisting[language=Matlab]{Codici/Cap6/potenze.m}

\noindent Nella seguente tabella \'e possibile visualizzare i risultati ottenuti: 

\begin{center}
	\begin{tabular}{ | l | c | c |}
		\hline
		$n$ & \textit{numero iterazioni effettuate} & \textit{stima autovalore} \\ \hline
		$100$ & $167$ & $7.8224$ \\
		$200$ & $420$ & $7.8803$ \\
		$300$ & $638$ & $7.8916$ \\
		$400$ & $721$ & $7.8949$ \\
		$500$ & $743$ & $7.8964$ \\
		$600$ & $824$ & $7.8974$ \\
		$700$ & $893$ & $7.8976$ \\
		$800$ & $868$ & $7.8967$ \\
		$900$ & $795$ & $7.8957$ \\
		$1000$ & $775$ & $7.8954$ \\
		\hline
	\end{tabular}
\end{center}